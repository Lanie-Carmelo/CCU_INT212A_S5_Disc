\documentclass[stu,12pt,floatsintext]{apa7}

% Language and citation setup
\usepackage[american]{babel}
\usepackage{csquotes}
\usepackage[style=apa,sortcites=true,sorting=nyt,backend=biber]{biblatex}
\DeclareLanguageMapping{american}{american-apa}
\addbibresource{references.bib}

% Font and encoding
\usepackage[T1]{fontenc}
\usepackage{newtxtext,newtxmath}  % Modern Times-like font with math support

% Document metadata
\title{Session 5 Discussion}
\author{Lanie Molinar}
\authorsaffiliations{Colorado Christian University}
\duedate{June 4, 2025}
\course{INT-212A}
\professor{Carrie Faust-Silva}

\begin{document}
\maketitle
\thispagestyle{plain}
\pagestyle{plain}

\section{Introduction}

A biblical worldview is a way of seeing the world that helps us understand who God is, who we are, and what our purpose in life is. It provides a framework for understanding our calling and how we are meant to live. Without this foundation, we may struggle to recognize our true identity, live according to God’s design, or become the people He created us to be. For me, understanding God’s design has helped me reframe my disabilities not as limitations, but as tools for service and innovation in the technology and accessibility fields.

\section{What is a Biblical Worldview?}

A biblical, or Christian, worldview is a way of seeing the world that is based on the teachings of the Bible. According to \textcite{slickWhatAreChristian2008}, these are some of the essential elements of a biblical worldview:
\begin{itemize}
    \item An absolute God exists, and He created the universe.
    \item We are created in God's image, and thus all people are worthy of respect and honor.
    \item Man was given dominion over creation by God.
    \item Mankind is fallen, and Jesus is our only hope for redemption.
    \item The Bible is the Word of God.
    \item God provides for His creation.
\end{itemize}

These elements form the foundation of a biblical worldview, shaping how we understand our identity, purpose, and the world around us. They help us see ourselves as part of God's creation, with inherent value and dignity, and guide us in living according to His design.

\section{How Does A Biblical Worldview Help Us Understand God's Design?}

A biblical worldview helps us understand God's design by showing us that we each have a purpose. Our calling is not random, self-generated, or even based on the desires of our family members—though they may wish for us to pursue certain paths. Instead, our calling is rooted in God's plan for us. As \textcite{Smith2024} explains, God's purposes are always in "deep congruity" with how He made the world and how He made each of us. Smith describes congruence as the alignment between who we are and what God calls us to do. This means that understanding our calling begins with knowing ourselves fully, as Paul encourages in \textcite[Romans 12:3]{Tyndale1996}. This aligns with \textcite[Jeremiah 1:5]{Tyndale1996}, where God affirms that His design for us is intentional and purposeful from the very beginning. Psalm 139:13–14 reinforces the idea that our identity and calling are not accidental but intricately woven by God.

When we view ourselves through the lens of Scripture, we understand that our gifts, passions, interests, and even our limitations are all part of God's design. Over time, I have come to see that my multiple disabilities—often viewed as limitations—can actually be a source of insight and strength in the technology field. This realization did not come all at once, but through many small moments of frustration and reflection. Each time I encountered an inaccessible tool or a barrier that others didn’t notice, I began to understand that my perspective was unique and valuable. These experiences have shaped my desire to create accessible tools that serve others, especially those with intersecting disabilities. As Paul writes in \textcite[2 Corinthians 12:9]{Tyndale1996}, “My grace is sufficient for you, for my power is made perfect in weakness.” This verse reminds me that God can use what the world sees as weakness to accomplish His purposes.

\textcite[chapter 3]{Smith2024} also encourages us to consider where we "feel the brokenness of the world," as this can be a strong indicator of our calling. For me, that brokenness is most visible in the barriers faced by disabled people—especially those who have multiple disabilities. While individuals with a single disability may find tools or support systems that meet their needs, those of us with intersecting disabilities often fall through the cracks. A tool that works well for someone who is blind may not be usable by someone who is also autistic or has chronic pain, and support organizations focused on one disability often do not know how to support someone with multiple disabilities. I experience these challenges daily—not just as inconveniences, but as signs that the world is not yet as it should be. Whether it is a website I cannot navigate with a screen reader or a tool that excludes users with cognitive differences, these moments remind me that accessibility is not a given. They also fuel my desire to be part of the solution. I believe God is calling me to use my perspective and experiences to help build a more inclusive world, especially through technology. This sense of purpose is not separate from who I am—it is deeply congruent with how God made me.

\section{Conclusion}
A biblical worldview is essential for understanding our calling and how we are meant to live. It provides a framework for recognizing our identity, purpose, and the unique ways we can contribute to God's design. As \textcite{Smith2024} emphasizes, our calling is not separate from who we are—it is deeply congruent with how God created us. By viewing ourselves through the lens of Scripture, we can see our gifts, passions, and even our limitations as part of God's plan. For me, this means embracing my lived experience with multiple disabilities as a source of strength, insight, and purpose. Rather than seeing these challenges as obstacles, I am learning to see them as part of the way God is equipping me to serve others. This understanding helps me align my life with His purposes and gives me hope that I can make a meaningful impact in the world, especially in the area of accessibility and inclusion. By embracing this calling, I am learning to see my challenges not as limitations, but as opportunities to reflect God's love and justice in a world that still has much room to grow. I hope my story encourages others to reflect on how their own challenges might be part of God’s unique design for their lives.

\printbibliography
\end{document}